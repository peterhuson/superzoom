%%%%%%%%%%%%%%%%%%%%%%%%%%%%%%%%%%%%%%%%%%%%%%%%%%%%%%%%%%%%%%%%%%%%%%%%%%%%%%%%%%%%%%%%%%%%%%%%
%
% CSCI 1430 Project Progress Report Template
%
% This is a LaTeX document. LaTeX is a markup language for producing documents.
% Your task is to answer the questions by filling out this document, then to 
% compile this into a PDF document. 
% You will then upload this PDF to `Gradescope' - the grading system that we will use. 
% Instructions for upload will follow soon.
%
% 
% TO COMPILE:
% > pdflatex thisfile.tex
%
% If you do not have LaTeX and need a LaTeX distribution:
% - Departmental machines have one installed.
% - Personal laptops (all common OS): http://www.latex-project.org/get/
%
% If you need help with LaTeX, come to office hours. Or, there is plenty of help online:
% https://en.wikibooks.org/wiki/LaTeX
%
% Good luck!
% James and the 1430 staff
%
%%%%%%%%%%%%%%%%%%%%%%%%%%%%%%%%%%%%%%%%%%%%%%%%%%%%%%%%%%%%%%%%%%%%%%%%%%%%%%%%%%%%%%%%%%%%%%%%
%
% How to include two graphics on the same line:
% 
% \includegraphics[width=0.49\linewidth]{yourgraphic1.png}
% \includegraphics[width=0.49\linewidth]{yourgraphic2.png}
%
% How to include equations:
%
% \begin{equation}
% y = mx+c
% \end{equation}
% 
%%%%%%%%%%%%%%%%%%%%%%%%%%%%%%%%%%%%%%%%%%%%%%%%%%%%%%%%%%%%%%%%%%%%%%%%%%%%%%%%%%%%%%%%%%%%%%%%

\documentclass[11pt]{article}

\usepackage[english]{babel}
\usepackage[utf8]{inputenc}
\usepackage[colorlinks = true,
            linkcolor = blue,
            urlcolor  = blue]{hyperref}
\usepackage[a4paper,margin=1.5in]{geometry}
\usepackage{stackengine,graphicx}
\usepackage{fancyhdr}
\setlength{\headheight}{15pt}
\usepackage{microtype}
\usepackage{times}
\usepackage{booktabs}
\usepackage{hyperref}

% From https://ctan.org/pkg/matlab-prettifier
\usepackage[numbered,framed]{matlab-prettifier}

\frenchspacing
\setlength{\parindent}{0cm} % Default is 15pt.
\setlength{\parskip}{0.3cm plus1mm minus1mm}

\pagestyle{fancy}
\fancyhf{}
\lhead{Final Project Progress Report}
\rhead{CSCI 1430}
\rfoot{\thepage}

\date{}

\title{\vspace{-1cm}Final Project Progress Report}


\begin{document}
\maketitle
\vspace{-3cm}
\thispagestyle{fancy}

\section*{Definitions}

\textbf{Team name: \emph{Super Zoom}}

\textbf{Team members: \emph{Michael Litt, Peter Huson, Gabe Weedon, Mary Dong}}

\textbf{TA name: \emph{Haoze Zhang}}

\section*{Project}
\begin{itemize}
  \item \textbf{What is your project idea?}
  
  Our project idea has been fleshed out to potentially follow one of the following avenues: 
  \begin{enumerate}
    \item Perform adversarial testing on the ProGanSR network to evaluate its ability to upsample in a-typical, corner cases. 
        \subitem We may explore expanding the training set of the network to improve performance on adversarial cases.  
    \item Evaluate traditional engineering-based methods for upsampling images and compare them to learning-based models. This will likely be very difficult.  
  \end{enumerate}
  
  
  \item \textbf{What data have you collected?}
  
  We've downloaded the DIV2K dataset, which was used to train and test the ProGanSR model. To ensure that we've reproduced and set up the model correctly, we are going to use run our reproduced version on this dataset (specifically images used as demos in the paper).
  
  In order to perform adversarial testing, we are also going to download other datasets and run ProGanSR on these sets. We're trying to find data that's quite different from DIV2K in order to challenge the NN on different types of images. We're still finding other interesting data, but so far we have:
  \begin{itemize}
      \item MNIST Dataset
      \item Document (text-based) images
      \item CelebA Dataset (faces)
      \item Custom images taken with a portrait-mode camera (some parts are blurred/LR; faces are sharp/HR)
      \item ImageNet
      \item Medical images
  \end{itemize}
  
  We definitely want some of these to be high-res images. However, we also want to try the model on already low-res images to see how it would perform.
  
  \item \textbf{What software have you built or used?}
  
  We have started to set up the ProGanSR network from their code that is on GitHub: \url{https://github.com/fperazzi/proSR}. This is the code we plan to use to attempt to `trick' the neural network.
  
  We will use the following \href{https://cloud.google.com/deep-learning-vm/docs/pytorch_start_instance}{PyTorch VM} to run the library on a GCP instance. 
  
  \item \textbf{What has each team member contributed thus far?}
  \begin{itemize}
      \item \textbf{Mary:} Read the ProGanSR paper, highlighted important parts and shared the annotated version with my group members; started reproducing the ProGanSR model and looked more into the DIV2K dataset.
      \item \textbf{Michael:} Read the ProGanSR paper, started looking into classical image upscaling, however implementing those seems less feasible than using a neural network approach. I think we should try using portrait mode images to test the network.
      \item \textbf{Peter:} Research into the ProGanSR codebase, exploring 2019 \href{https://competitions.codalab.org/competitions/21439#results}{NTIRE} results to potentially find a more state-of the art solution. Exploring how we will train on GCP using \href{https://github.com/fperazzi/proSR/issues/22}{different GPUs} 
      \item \textbf{Gabe:} Looked into PyTorch compared to TensorFlow and getting it running on GCP. Read over the research paper for ProGanSR. 
  \end{itemize}
  
  
  \item \textbf{What intermediate results have you generated?}
  
  We have read a few papers in this category of research and found that learning-based approaches will be most reasonable to implement in the short time we have. We began working on implementing the ProGanRS network using the released code (see above), and have hit a few bugs installing the codebase in PyTorch. 
  
  \item \textbf{What problems have you faced or still have to consider?}
  
  The ProGanSR network has open source code available that uses the PyTorch machine learning library, not TensorFlow which we are more familiar with. We are looking into finding resources to teach ourselves PyTorch and getting PyTorch working on the GCP virtual machines. 
  
  The paper describes a training time of 1000 minutes, which would be very expensive if we ran this training on a GCP instance. As such, we need to consider training time and compute power when evaluating what is feasible to accomplish given our cost and time constraints. We have considered using the \href{https://github.com/fperazzi/proSR/issues/1}{pretrained model} provided by the researchers if training the network ourselves is not an option. 
  
  \item \textbf{Is there anything that we can do to help? E.G., resources, equipment.}
  
  If any TA's are familiar with PyTorch and know of good learning resources, that could be very useful.
  
 
\end{itemize}


\end{document}